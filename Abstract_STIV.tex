%% 
%% Copyright 2007-2020 Elsevier Ltd
%% 
%% This is a modified file from the 'Elsarticle Bundle', which is available under the conditions of the LaTeX Project Public
%% License, either version 1.2 of this license or any later version.
%% ---------------------------------------------
\documentclass[preprint,12pt]{elsarticle}

%% The amssymb package provides various useful mathematical symbols
\usepackage{amssymb}

%% The lineno packages adds line numbers.
\usepackage{lineno}

\journal{EWRI}

\begin{document}

\begin{frontmatter}

\title{Improving Space-Time Image Velocimetry Accuracy with Deep Learning: Integrating Synthetic and Real-World Data}

\author[inst1]{Tenorio, A}
\author[inst1]{Fernández, R}
\author[inst2]{Engel, F}
\author[inst1]{Liu, X}

\affiliation[inst1]{organization={Pennsylvania State University}, 
			department={Department of Civil and Environmental Engineering},
            city={University Park},
            state={Pennsylvania},
            country={USA}}
            
\affiliation[inst2]{organization={US Geological Survey}, 
			department={Observing Systems Division},
            city={San Antonio},
            state={Texas},
            country={USA}}

\begin{keyword}
STIV \sep Deep Learning \sep Flow measurements
\end{keyword}

\end{frontmatter}

\linenumbers

Quantifying streamflow is essential for resource management, habitat monitoring, and emergency response, but extreme events pose challenges for direct measurements due to safety and accessibility concerns. Remote sensing techniques, such as Space-Time Image Velocimetry (STIV), allow for non-contact measurements of velocity and discharge. STIV uses video footage of the water surface to generate Space-Time Images (STI), assuming surface textures will act as passive tracers. However, environmental conditions such as sunlight conditions, surface reflections or rain can hinder the quality of the STIs. While Wavenumber–Frequency Spectra (WFS) filters can improve STI quality, they struggle in real-time applications. Incorporating deep learning can enhance real-time accuracy, as demonstrated in previous work with synthetic STIs. We aim to extend this by combining synthetic, computational, laboratory-controlled, and real-world data to improve the accuracy of flow measurements in natural settings.

\end{document}