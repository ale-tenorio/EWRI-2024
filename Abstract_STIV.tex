%% 
%% Copyright 2007-2020 Elsevier Ltd
%% 
%% This is a modified file from the 'Elsarticle Bundle', which is available under the conditions of the LaTeX Project Public
%% License, either version 1.2 of this license or any later version.
%% ---------------------------------------------
\documentclass[preprint,12pt]{elsarticle}

%% The amssymb package provides various useful mathematical symbols
\usepackage{amssymb}

%% The lineno packages adds line numbers.
\usepackage{lineno}

\journal{EWRI}

\begin{document}

\begin{frontmatter}

\title{Improving STIV Accuracy with Deep Learning: Integrating Synthetic and Real-World Data}

\begin{keyword}
STIV \sep Deep Learning \sep Flow measurements
\end{keyword}

\end{frontmatter}

\linenumbers

Quantifying streamflow is vital for resource management, habitat monitoring, and emergency response. Streamflow measurements typically use direct or indirect approaches that depend on site and flow conditions. Extreme events become difficult to measure due to the need for timely site access, sensor maintenance issues, and safety concerns. For this reason, remote sensing (non-contact) methods using imagery are becoming more prevalent due to their reliability, cost and safety of use.
\\
Non-contact methods like Space-time Image Velocimetry (STIV) provide an opportunity to gather direct measurements of velocity and discharge. With a video recording of the water surface, the pixels along a search line at consecutive time instances can be stacked and form a Space-Time Image (STI). The fundamental assumption behind the technique is that visible texture on the water surface acts as a passive tracer relative to the surface flow (Fujita, I., et al., 2021). The disturbance along the line will be advected with an inclined angle $\phi$ in the STI corresponding to the advection velocity.
\\
The visible texture assumption does not always occur in nature. Sunlight conditions, surface reflections, rain, lens obstructions and camera positioning might pose challenges to obtain a good quality STI. Wavenumber–Frequency Spectra (WFS) based filters can already be utilized to improve the quality of the STI (Fujita, I., 2020), but these filters are not yet robust in a real-time measurement environment.
\\
A more suitable strategy for a real-time measurement system is to incorporate the deep learning method (Watanabe, K., et al., 2021). This approach can continuously improve performance by learning additional data. While Watanabe et al. (2021) demonstrated significant advancements using synthetic STIs, we aim to extend this approach by incorporating synthetic STIs, computational models, lab-produced data, and real-world videos to enhance accuracy on natural scenarios.

\end{document}